\begin{itemize}
	\item Definition of a task tree: type of task model which describe user actions
    	\begin{itemize}
		\item  e.g. used in Goals, Operators, Methods and Selection Rules (GOMS), TaskMODL and ConcurTastTrees (Zitat?)
	\end{itemize}
	\item Aim: analyze and compare effective and expected user interactions, semi-automatic usability evaluation
	\item two possible variants of task tree generation 
 	\begin{itemize}
		\item - generated from expected actions at design time
      		\begin{itemize}
			\item Tools/Software: Critique, ReverseAllUI (based on models of GUI) (sources...)
      			\item Problem: simplified or complete task tree
		\end{itemize}
		\item Actual user interactions (e.g. from event driven software)
     		\begin{itemize} 
			\item Programming by example: effective user actions recorded
			\item Problem: usability improved only on a local area
		\end{itemize}
	\end{itemize}
	\item Harms tries to get a more global view on software: usability could be improved by that (cite) \citep{harms2013}
	\item Disadvantages from harms method here!
	\item In this work I will cover the following aspects...
\end{itemize}


\section{Terminology}
\begin{itemize} 
	\item Components of a task tree: (with picture)
	\begin{itemize}
		\item root node: represents the overall task which contains all subtasks, the user wants to "reach" this node by his actions/his input
		\begin{itemize}
    			\item Order something in a webshop
		\end{itemize}
		\item intermediate nodes: subtasks which are steps towards the overall task
		\begin{itemize}
			\item e.g.: create a new account/log in, put something into the basket
		\end{itemize}
		\item leaf nodes: actions (e.g. click, scroll, textinput) which cause an event (e.g. \textit{onclick} (click) or \textit{onchange} (textinput)
		\begin{itemize}
			\item e.g.: enter a product name in a text field and click on the "search"-button
		\end{itemize}
	\end{itemize}
	\item Where should i write about the difference between tasks and taskinstances? 	
	\item temporal relationship: the order of the executed actions is important, so it is saved in the task tree
	\item different ways of temporal relationship used by harms: 
	\begin{itemize}
		\item sequences: children executed in specific order
		\item iteration: only one child, executed zero or more times
		\item selection: only one of the children is executed
		\item optional:  only one child, executed once or not
	\end{itemize}	
	\item Trace: recorded sequence of events: monitoring module records events caused by user actions, stored 
	\item add example of a trace (maybe this goes to case study)
\end{itemize}
\section{Trace Based Task Tree Generation by Harms}
%Basic approach: 4 possible designs, 3 of them used in the task trees (?)
%  - conceptual design: describes types of entities/objects which are to be edited with a software (umformulieren!) and their relationship towards each other
%    -> not used in task tree generation
%  - semantic design: defines functions of the conceptual design
%
%    -> complies with root node: one task tree is build per function (overall task)
%  - syntactical design: instruction to execute the funktions of the semantic design
%    -> complies with intermediate note: subtasks, temporal relationship
%  - lexical design: necessary physical actions to execute the syntactical instructions
%    -> complies with leaf nodes (comply with event tasks: no children, not defining temporal relationship): actions
\begin{itemize}
	
	\item Procedure: start with leaf nodes to creat a task tree
 	\item for every event in a trace: crate an event task instance
  	\item event tasks instances stored in recorded order

	\item Iteration detection:
	\begin{itemize} 
		\item identical tasks, which often repeat directly (e.g.: click on the same button a few times): Iteration
		\item generate a new task model of tyoe iteration: iterated event task as single child
		\item replace every occured iteration of the event task with the iteration task node 
	\end{itemize}

	\item Sequence detection:
	\begin{itemize}
		\item task list scanned for identical subsequences
    		\item most occured and longest subsequence: propably a logical and useful subtask
		\item new task node type sequence generated
		\item every identical subsequence replaced with the new task node 
		\item if same length and count of subsequence: first subsequence will be replaced (always only one subsequence replaced!
    		\item minimal length of a subsequence: 2 actions/ event tasks
	\end{itemize}
	\item Repetition of Detections:
	\begin{itemize}
		\item iteration and sequence detection repeatet until there are no matches anymore
  		\item replaced sequences may already contain iterations/sequences 
		\item in the end: list with task trees and event tasks that does not fit in as an iteration or sequence
		\item Problem: some event tasks are used only once (e.g. login) or very seldom by one user
		\item solution: compare data of more users 
	\end{itemize}
\end{itemize}
