\section{Improving Task Tree Generation}

Find similar sequences with bioinformatics algorithms
 
\section{Alignments}
\textit{Maybe this should go to introduction?}
What are alignments? 
\begin{itemize}
	\item String algorithm often used in bioinformatics 
	\item arranging sequences of DNA, RNA, Peptids to identify meaningful regions.
	\item Meaningful regions are conserved, little mutations in sequences
	\item Representation: Sequences in rows, aligned 
	\item Minimal example of two aligned sequences here (with gaps though)
	\item 2 Main categories: global and local alignments 
	\item global: find best alignment end to end: better for very similar sequences
	\item local: find best alignment of subsequence: better for finding conserved regions of not so similar sequences
	\item Basic Algorithms: NeedleMan Wunsch(global), Smith Waterman(local) 
	\item 
\end{itemize} 
\section{Substitution matrix}
\begin{itemize}
	\item Representation of how similar two elements in a sequence are. 
	\item Alignment algorithms use this information to arrange the two sequences 
	\item Usually symmetric matrices where each cell represents how good or bad it is to substitute element a with element b 
\end{itemize}
\subsection{Biological background}
In Biology popular matrices are generated from real DNA data (PAM, BLOSUM) (cite)
\subsection{Objectdistance Substitution Matrix}
\textit{Not sure if this belongs here, I need all the vocabulary  like eventtaskinstaces, Task}
\begin{itemize}
	\item The substitution used here is based on the distance of elements in the GUI model of events
	\item What is a GUI Model? Small example and description here
	\item Algorithm of creation: For eventtaskinstances:  substition score of (a,b) = distanceInGuiModel(a,b) 
\end{itemize}

\section{Smith Waterman Algorithm For Repeated Matches}
\begin{itemize}
	\item Modified Version of the smith waterman algorithm
	\item Finds all local alignments of two sequences that reach a threshold score, not just the best
	\item Uses dynamic programming
	\item Exact method, not heuristic
	\item Two sequences: 
	\item y: pattern
	\item x: sequence in which we search all subsequences of y repeatedly
	\item Final aligment: x has matched and unmatched regions
	\item Uses matrix (dynamica programming matrix) where each cell stores the best score for aligning the elements at the position the cell represents
	\item Advantage: Possible alignments with less score do not need to be calculated recursively
	\item Score of a matched region: sum of all scores for each position minus the score threshold 
	\item first row has a different meaning than in basic Smith waterman (score of the last matched subsequence)
	\item Definitions:
	\item F is the dynamica programming matrix
	\item s is the substition matrix
	\item d is the gap penalty
\end{itemize}

\subsection{Intialization}:
	\begin{itemize}
		\item \[F(0,0) = 0\]
%		\item Fill first col with 0, 
	\end{itemize}

\subsection{Recursion}
	\begin{itemize}
		\item Build dynamic programmic matrix
		\item  \[F(i,0) = max \left\{ \begin{array}{lr}F(i-1,0)&\\F(i-1,j)-T& j=1,\dots,m\end{array}\right. \]
		\item  \[F(i,j) = max \left\{ \begin{array}{lr}F(i,0),\\F(i-1,j-1)+s(x_i,y_i),\\F(i-1,j)-d,\\F(i,j-1)-d\end{array}\right.\]
		\item Explain each choice we can take here
		\item Store every choice we took
	\end{itemize}
\subsection{Traceback}
Go back from the addition cell in the first row to first cell to get alignment

\subsection{Match retrival}
	\begin{itemize}
		\item Search and store  each match that scored more than the threshold
		\item Due the nature of this algoritm, all matches that can be found have this minimal score
	\end{itemize}

\section{Task Tree Generation}


