\section{Improving Task Tree Generation}
Find similar sequences with bioinformatics algorithms

\section{Alignments}
\textit{Maybe this should go to introduction?}
What are alignments? 
\begin{itemize}
	\item String algorithm often used in bioinformatics 
	\item arranging sequences of DNA, RNA, Peptids to identify meaningful regions.
	\item Meaningful regions are conserved, little mutations in sequences
	\item Representation: Sequences in rows, aligned 
	\item 2 Main categories: global local alignments 
	\item global: find best alignment end to end: better for very similar sequences
	\item local: find best alignment of subsequence: better for finding conserved regions of not so similar sequences
	\iten Basic Algorithms: NeedleMan Wunsch(global), Smith Waterman(local) \(just cite them\)
\end{itemize} 
\section{Substitution matrix}
\begin{itemize}
	\item Representation of how similar two elements in a sequence are. 
	\item Alignment algorithms use this information to arrange the two sequences 
	\item Usually symmetric matrices where each cell represents how good or bad it is to substitute element a with element b 
\end{itemize}
\subsection{Biological background}
In Biology popular matrices are generated from real DNA data \(PAM, BLOSUM\) \(cite\)
\subsection{Objectdistance Substitution Matrix}
\textit{Not sure if this belongs here, i need all the vocabulary  like eventtaskinstaces, Task
\begin{itemize}
	\item The substitution used here is based on the distance of elements in the GUI model of events
	\item What is a GUI Model? Small example and description here
	\item Algorithm of creation: For eventtaskinstances:  substition score of \(a,b\) = distanceInGuiModel\(a,b\) 
\end{itemize}

\section{Smith Waterman Algorithm with repeated matches}
Finds all local alignments of two sequences that reach a threshold score


\subsection{Intialization}:
Fill first col with 0, first row has a different meaning in this algorithm (score of the last matched subsequence)

\subsection{Recursion}
Build dynamic programmic matrix: Fill each field with this again sligtly modified recursion formula

\subsection{Traceback}
Go back from the addition cell in the first row to first cell to get alignment


