Usability has become a very important aspect of software development. 
Software is the interface between humans and computers. 
Once an interface is designed badly users tend to use the software unefficiently or stop using it at all due to frustrating experiences.
Therefore it is critical to test the usability during design of the software but also evaluate the feedback given from users. 
This feedback could either be created manually by the user e.g. by filling out a feedback form or it could be gathered automatically by observing how users actually use the software.
In this work we will improve existing methods for evaluating collected user interactions.

Task trees are a method to describe and structure the process of users interacting with software. 
The designer of the software can generate a task tree from the GUI model e.g. from HTML structure. 
In contradiction to the task trees generated at design time they can also be derived from recorded user actions. 
In this case, task trees offer the possibility to semi-automatically evaluate the usability of the software since it is then possible to compare expected with the actual user behaviour \citep{harms2013}.  

\begin{itemize}
	\item Definition of a task tree: type of task model which describe user actions
    	\begin{itemize}
		\item  e.g. used in Goals, Operators, Methods and Selection Rules (GOMS), TaskMODL and ConcurTastTrees (Zitat?)
	\end{itemize}
	\item Aim: analyze and compare effective and expected user interactions, semi-automatic usability evaluation
	\item two possible variants of task tree generation 
 	\begin{itemize}
		\item - generated from expected actions at design time
      		\begin{itemize}
			\item Tools/Software: Critique, ReverseAllUI (based on models of GUI) (sources...)
     			\item Problem: simplified or complete task tree
		\end{itemize}
		\item Actual user interactions (e.g. from event driven software)
     		\begin{itemize} 
			\item Programming by example: effective user actions recorded
			\item Problem: usability improved only on a local area
		\end{itemize}
	\end{itemize}
	\item Harms tries to get a more global view on software: usability could be improved by that (cite) 
	\item Disadvantages from harms method here!
	\item In this work I will cover the following aspects...
\end{itemize}


