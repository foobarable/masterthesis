Usability has become a very important aspect of software development. 
Once an interface is designed badly users tend to use the software inefficiently or stop using it at all due to frustrating experiences.
Therefore, it is critical to test the usability during software design but also evaluate the feedback given from users. 
This feedback can be gathered automatically by observing how users actually use the software.
In this work we will improve existing methods for evaluating collected user interactions.

Task trees are one well method to describe and structure the process of users interacting with software. 
We define them as a type of task model which desribes user actions. There are two possibilities to create a task tree.
First of all, the designer of the software can generate a task tree at design time to model the expected user behaviour\cite{harms2013}.
The second option is to generate the task trees based on observed user interactions. 
Those task trees represent the effective user behaviour.
A comparison of the effective and the expected user behaviour offers the possibility to semi-automatically evaluate the usability of the software.  

The method of task trees to model tasks is used in Goals, Operators, Methods and Selection Rules (GOMS),


\begin{itemize}
	\item  e.g. used in TaskMODL and ConcurTastTrees (Zitat?)
 	\begin{itemize}
		\item - generated from expected actions at design time
      		\begin{itemize}
			\item Tools/Software: Critique, ReverseAllUI (based on models of GUI) (sources...)
     			\item Problem: simplified or complete task tree
		\end{itemize}
		\item Actual user interactions (e.g. from event driven software)
     		\begin{itemize} 
			\item Programming by example: effective user actions recorded
			\item Problem: usability improved only on a local area
		\end{itemize}
	\end{itemize}
	\item Harms tries to get a more global view on software: usability could be improved by that (cite) 
\end{itemize}


