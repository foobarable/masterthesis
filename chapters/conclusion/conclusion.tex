With the alignment approach for sequence detection we created a valuable framework for task tree generation. 
It is able to detect common interactions with a software of multiple users, to weight the importance of each found interaction and to create a task tree that represents
the users behaviour.
For the creation of the task tree we use methods adopted from bioinformatics that are able to perform an approximate comparison of sequences of tasks. 
This alignment method is well configurable by several parameters that influence the quality and amount of created task trees.
Disadvantages of this approach are performance and resource intensity issues as well as a missing merging of similar task trees.

Further research should focus on the evaluation of the alignment and substitution matrix parameters. 
It is conceivable that those parameters can be automatically learned or adjusted to the processed data.
This may allow to find good results on any kind of data. 
For the learning of the parameters we would need to have a definition of what a ideal task tree for a specific task looks like.
This leads to the overall question what a good task is in the perspective of usability.
If such a definition existed, one could optimize the parameters to create a task tree that converges to the ideal task tree.

In summary we can say that once the automatic adjustment of the alignment and substitution matrix parameters is researched the overall goal to have a method for semi-automatic usability evaluation can be achieved.



