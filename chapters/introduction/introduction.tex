Usability has become a very important aspect of software development. 
Once an interface is designed badly users tend to use the software inefficiently or stop using it at all due to frustrating experiences.
Therefore, it is critical to test the usability during software design but also evaluate the feedback given from users. 
This feedback can be gathered automatically by observing how users actually use the software.
In this work we will improve existing methods for evaluating collected user interactions.

Task trees are one well method to describe and structure the process of users interacting with software. 
We define them as a type of task model which describes user actions. 
There are two possibilities to create a task tree.
First of all, the designer of the software can generate a task tree at design time to model the expected user behaviour\cite{harms2013}.
This approach is used by ConcurTaskTrees\cite{paterno2003}, a notation for task model specification and ReverseAllUI\cite{bandelloni2008}. 
With those methods it is possible to extract task models from several markup languages like HTML, XHTML or TERESA XML.
The second option is to generate the task trees based on observed user interactions. 
Those task trees represent the effective user behaviour. 

There are several approaches to generate task models automatically from user traces e.g. CRITIQUE\cite{Hudson1999}, which creates GOMS (Goals, Operators, Methods and Selection rules) models.
Harms et al. described a method to create task trees from observed user interactions\cite{harms2013}.
A comparison of the effective and the expected user behaviour offers the possibility to semi-automatically evaluate the usability of the software.  

In this work we try to improve the approach from Harms et al. by applying methods usually common in bioinformatics. 
We will first cover the foundations needed in the following chapters and then describe our approach in chapter \ref{chap:approach}. 
After that, we will give some details about the implementation of our approach. 
In chapter \ref{chap:casestudy} we will evaluate our method by applying it on real data and discuss our results.

