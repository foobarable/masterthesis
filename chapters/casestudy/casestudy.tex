To validate our method we will run the alignment approach of task tree generation against the same case study as Harms et al. did.
The data was collected on an application portal of the university. Figure \ref{fig:screenshotmasterportal} shows a screenshot of the first page of the portal.
After login, users can fill out multiple forms regarding their personal data as well as upload their CVs. The data entered was fully anonymized. 
In this case study 555 user created arround 3635 feasible user sessions. Further details are enlisted in table \ref{tab:casestudy2}.



\begin{table}
	\centering
	 \begin{tabular}{|r|c|}
		   \hline
		   & \textbf{Application Portal }\\
		     \hline
		       Start of Recording & 25 October 2013 \\
		       End of Recordning & 7 March 2014 \\
		       Recorded Users & 555 \\
		       Recorded User Sessions & 4,129 \\
		       Considered User Sessions & 3,602 \\
		       \hline
		         \textbf{Recorded Events} & 350,368 \\
		         Relevant Events & 306,568 \\
		         Double Clicks & 6,437 \\
		         Focus Changes & 89,825 \\
		         \hline
			   \textbf{Considered Events} & 210,306 \\
			   Different Events & 1,897 \\
			   \hline
			    \end{tabular}
			    \caption{Case study overview}
	\label{tab:casestudy2}
\end{table}



Another aspect of our approach we want to examine is the necessity of the calculation of the distance between non-event-tasks. 
The calculations of those distances are very expensive operations so we are interested if this step could be left out and we still find approximatly the same amount of tasks, in the same quality.


\begin{table}
	\begin{tabularx}{\textwidth}{|c|X|c|c|}
	   \hline
		   Parameter & Function & Definition & Used values\\
	     \hline
	       k & The value of the maximum score in the substitution matrix& \ref{def:scorewithmaximalscore}& 6 \\
	       L & Penalty for the score between non-event-tasks & \ref{def:scoreadjusted} & 3\\
	       T & Threshold score for determination of match importance & \ref{def:treshold} &9\\
	       g & Gap penalty for inserting gaps & \ref{def:gappenalty} &3\\
	       f & Number of occurrences a task must at least have in all user sessions to be replaced & \ref{def:minoccurrencecount} &3\\
	       \hline
 \end{tabularx}
 \caption{Table of all parameters of the sequence detection.}
 \label{tab:parameters}
 \end{table}


\begin{itemize}
	
	\item Two case studies: 
		\begin{itemize}
			\item Application portal case study. 
			\item Website of research group
		\end{itemize}
	\item Table of Case studies
	\item Screenshot of Application Master Portal
	\item Screenshot of Website of research group
	\item Show XML recorded traces
	\item When running the full case study alot of perfomance problems came up.
	\item Memory consumption and computation time grew alot due to the nature of the algorithm
	\item Computation moved from a Intel(R) Core(TM) i5-2520M CPU @ 2.50GHz with 8GB Ram to a AMD Opteron(TM) Processor 6276 (64 core) with 250GB Ram.
	\item The java virtual machine was given 64GB
	\item Still computation time extremly large (see performance evaluation)
	
	\item Maybe change stop criterium: 10% of the matches found in the first run
	\item Other solution: Update substution matrix, so non-Event tasks can have a negative score (less likely to find matches then)
\end{itemize} 

\section{Data Preprocessing}
After loading the input data from all XML files the following preprocessing steps have been performed. The information about each command is copied from its man page in AutoQUEST.
\begin{itemize}
	\item condenseHTMLGUIModel: Merges all equal nodes in the GUI-Model.
	\item condenseMouseClicks: Reduces a sequence of mouse button down, mouse button up and mouse click with the same button on the same event target to a single mouse click with that button on that target. The mouse button down and mouse button up events are discarded.
	\item correctKeyInteractionTargets: Iterates the provided sequences and sets the target of all key interaction events to the GUI element having the current keyboard focus. The current keyboard focus is determined either by keyboard focus events or by using the target of the first key interaction in a sequence. Events changing the keyboard focus are discarded herewith.
	\item correctTabKeyNavigationOrder: Iterates the provided sequences and corrects the order of events in case of tab key navigation. This is required, as from time to time the event of pressing the tab key for navigation in formulars comes before the text input event in a text input field out of which the tab key navigates.
\end{itemize}

\section{Calculation of distances between non-event-tasks}
The calculation of distances between non-event-tasks is unproblematic on very small subsets(ca.40 user sessions) of the case study.
But once we included more sequences the index for the array we use for storing the scores in the substitution matrix hit the limit of Integer.MAX\_VALUE.
A possible solution for this issue is not to calculate the distances between the non-event-tasks and thus saving computation time and ram usage.
In this section we investigate if the calculation between non-event-tasks is neccessary by comparing the two generated task trees, one with the additional calculations and one where 
we set every score involving a non-event-task to zero. Table \ref{tab:resultsnoneventtasks} shows the results of this experiment.
The number of found tasks is significantly higher when the distances between non-event-tasks are calculated. The time to generate the task tree increases drastically as well.
The quality of the generated task trees also differs. Without the calculated distances the generated task trees have a flatter structure and are shorter. 
With the calculation of the distances long interactions e.g. login or account creation procedures can be found. 
Those long tasks do not always represent correct user interactions. 
Figure \ref{fig:noneventaccountcreation} shows a task tree for the first part of the account creation process. 
We can see several subtasks of the task where the subtask does not model user behaviour correctly.
\texttt{Selection 2691} followed by by \texttt{optionality 2692} makes no sense since the text input of the optionality should just happen after the text input field has been clicked on. 
This happens in \texttt{iteration 1288}, the first child of \texttt{selection 2691}. So the optionality with the text input should actually be in a sequence with \texttt{iteration 1288} as its preceeding element.
Another thing that is no real user behaviour in this example task is \texttt{sequence 1656}. 
The first selection in it gives the posibility to either click or double click on an input field. 
While double clicking a text field is not an effective behaviour of a user, it is still a valid action to achieve his goal to enter his email address.
The meaningless part is the text input on the email input field, followed by the same event again. This should either be just one event at all or be found in an iteration.
There are several more examples where the non-event-task distance calculation did not improve the task tree quality.
In summary, the amount of tasks created  but not their quality can be increased by enabling the distance computing.

With the large increase of computational time in mind we will set the score to or from non-event-tasks to zero in all further experiments.
The reason for this is that we have an increase in time by over 3800\% even in this very small example and we cannot asume a linear scaling of this increase because nearly all algorithms we use scale have a complexity of $O(n^2)$.

\begin{figure}[h]
	\centering
	%\includegraphics[scale=0.7]{chapters/casestudy/noneventcreateaccount.png}
	\includegraphics[width=\textwidth]{chapters/casestudy/noneventcreateaccount.png}
	\caption{An example for a task found by alignment task tree generation with distance calculation between non-event-tasks.}
	\label{fig:noneventaccountcreation}
\end{figure}

\begin{table}[h]
	\centering
	\begin{tabular}{|c|c|c|}
		\hline
		& \textbf{with non-event-task distances}& \textbf{without non-event-task distances} \\
		\hline
		Sequences & 352 & 50 \\
		
		Iterations& 38  & 38 \\
		Selections& 328 & 21 \\
		Optionals & 6   & 8  \\
		\hline
		Time      &  72.2s & 1.9s \\
		Number of iterations & 35 & 7\\
		\hline
	\end{tabular}
	\caption{Results of the version with and without calculation of the distances between non-event-tasks}
	\label{tab:resultsnoneventtasks}
\end{table}



\section{Evaluation of termination conditions}
\begin{itemize}
	\item Run CTA with full case study
	\item show matches found per iteration and time per iteration

	\begin{table}
		\centering
	\begin{tabular}{|c|r|c|c|}
		  \hline
		  Iteration No. & Matches & Absolute time& Relative time \\
		  \hline
     		    0  & 1,112,794 & 26.17 & 26.17\\
	            1  & 381,190   & 39.16 & 12.99\\
		    2  & 167,677   & 45.52 & 6.36\\
		    3  & 63,964    & 51.00 & 5.48\\
		    4  & 26,171    & 55.83 & 4.83\\
		    5  & 12,627    & 60.72 & 4.89\\
		    6  & 8,660     & 65.48 & 4.76\\
		    7  & 7,517     & 70.34 & 4.86\\
		    8  & 7,232     & 75.02 & 4.68\\
		    9  & 7,214     & 79.74 & 4.72\\
		    10 & 7,214     & 84.49 & 4.75\\
		  \hline
		   \end{tabular}
		   \caption{Matches found per iteration and the time for each iteration. All times in minutes.}
		   \label{tab:timesandmatchesperiteration}
	\end{table}

	\begin{figure}
		\includegraphics[width=\textwidth]{chapters/casestudy/hasehase.png}
		\caption{}
		\label{fig:hasehase}
	\end{figure}
\end{itemize}


\section{Performance Evaluation}
\begin{itemize}
	\item Compare Core-TaskTrees(CT) with Core-TaskTrees-Algignment (CTA)
	\begin{itemize}
		\item Run CT with 10, 100, 1000, all user sessions on CS1
		\item Run CTA with 10, 100, 1000, all user sessions on CS1
		\item Show percentage of each step of CTA of all user sessions
		\item Run CT with 10, 100, 1000, all user sessions on CS2
		\item Run CTA with 10, 100, 1000, all user sessions on CS2
		\item Show percentage of each step of CTA of all user sessions
	\end{itemize}
	\item Alot of tables (which step of the algorithm takes how long, for each dataset
\end{itemize}

\section{Generated Task Trees}
\begin{table}
	\centering
 \begin{tabular}{|r|c|c|}
	   \hline
	      & \textbf{Harms et al.} & \textbf{Alignment approach} \\
	     \hline
	       \textbf{Generated Tasks} & 10,634 & 6,532 \\
	       Sequences & 9,530 & 2,759 \\
	       Iterations & 1,104 & 619 \\
	       Selections & -& 1,156 \\
	       Optionals & -& 101 \\
	       \hline
\end{tabular}
\end{table}


\begin{itemize}
	\item Show some meaningful generated task trees
	\item Show also tasktrees that are not useful
\begin{figure}
	\centering
	\includegraphics[]{chapters/casestudy/mixedtasktree.png}
	\caption{}
	\label{}
\end{figure}
\begin{figure}
	\centering
	\includegraphics[]{chapters/casestudy/newpassword.png}
	\caption{}
	\label{}
\end{figure}
\begin{figure}
	\centering
	\includegraphics[]{chapters/casestudy/preprocessing_needed.png}
	\caption{}
	\label{}
\end{figure}
\begin{figure}
	\centering
	\includegraphics[]{chapters/casestudy/newpassword-1.png}
	\caption{}
	\label{}
\end{figure}
\begin{figure}
	\centering
	\includegraphics[]{chapters/casestudy/login_process_repeated.png}
	\caption{}
	\label{}
\end{figure}
%\begin{figure}
%	\centering
%	\includegraphics[]{}
%	\caption{}
%	\label{}
%\end{figure}
\end{itemize}


\section{Discussion}
\begin{itemize}
	\item Task Graph
\end{itemize}
