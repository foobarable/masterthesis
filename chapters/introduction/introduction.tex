Related Work, Work of Pharms, Goal: Usability Evaluation \citep{harms2013}
Task trees what are they good for, Goal of the thesis
\section{Task Trees}
hier kommen neue dinge hin
What are task trees? Terminolog? Terminology used by Pharms
\section{Trace Based Task Tree Generation}
How does Pharms generate Task trees? What is the disadvantage? (tl,dr: Just exact sequence matching)

\section{Introduction}
Definition of a task tree: variant of task models which describe user actions
    -> e.g. used in Goals, Operators, Methods and Selection Rules (GOMS), TaskMODL and ConcurTastTrees (Zitat?)
Aim: analyze and compare effective and expected user interactions, semi-automatic (add more details)
2 possible variants of task trees based on: 
  - expected actions (e.g. with HTML (more details...))
      -> Tools/Software: Critique, ReverseAllUI (based on models of GUI) (sources...)
      -> Problem: simplified or complete task tree
  - effective actions (e.g. event driven software)
      -> Programming by example: effective user actions recorded, next steps can be supposed for the user
      -> Problem: usability improved only on a local area
=> global view on software: usability can be improved, software can be optimized 
Mention Paper
Structure of the thesis

\subsection{Terminology based on the paper}
Components of a task tree: (with picture)
  - root node: represents the overall task which contains all subtasks, is the goal/aim of the user action
    -> e.g.: order something online
  - intermediate nodes: subtasks which are steps towards the overall task
    -> e.g.: create a new account/log in, but something in the basket
  - leaf nodes: actions (e.g. click, scroll, textinput) which cause an event (e.g. onclick (click) or onchange (textinput)
    -> e.g.: enter a product name in a text field and click on the "search"-button
temporal relationship: the order of the executed actions is important, so it is saved in the task tree
  - different ways to work with temporal relationship; used by patrick: 
    -> sequences: children executed in specific order
    -> iteration: only one child, executed once or not
    -> selection: only one of the children is executed
Trace: recorded sequence of events
  - monitoring module records events caused by user actions, stored 
  - add example (table)

\subsection{Trace Based Task Tree Generation}
oriented to the paper "Trace based task tree generation"
Basic approach: 4 possible designs, 3 of them used in the task trees (?)
  - conceptual design: describes types of entities/objects which are to be edited with a software (umformulieren!) and their relationship towards each other
    -> not used in task tree generation
  - semantic design: defines functions of the conceptual design
    -> complies with root node: one task tree is build per function (overall task)
  - syntactical design: instruction to execute the funktions of the semantic design
    -> complies with intermediate note: subtasks, temporal relationship
  - lexical design: necessary physical actions to execute the syntactical instructions
    -> complies with leaf nodes (comply with event tasks: no children, not defining temporal relationship): actions
Procedure: start with leaf nodes to creat a task tree
  -> for every event in a trace: crate an event task
  -> event tasks stored in recorded order
Iteration detection:
  - identical tasks, which often repeat directly (e.g.: click on the same button a few times): Iteration
    -> generate a new task mode of tyoe iteration: iterated event task as single child
  - replace every occured iteration of the event task with the iteration task node (umformulieren?)
Sequence detection:
  - task list scanned for identical subsequences
    -> most occured and longest subsequence: propably a logical and useful subtask
      -> new task node type sequence generated
    -> every identical subsequence replaced with the new task node 
    -> if same length and count of subsequence: first subsequence will be replaced (always only one subsequence replaced!
    -> minimal length of a subsequence: 2 actions/ event tasks
Repetition of Detections:
  - iteration and sequence detection repeatet until their are no matches anymore
  - replaced sequences may already contain iterations/sequences 
  - in the end: list with task trees and event tasks that does not fit in as an iteration or sequence
  - Problem: some event tasks are used only once (e.g. login) or very seldom by one user
    -> solution: compare data of more users -> no logical event tasks will be lost
